% !TeX root = ../main.tex
\chapter{绪论}
\echapter{Introductions}

\section{背景}
\esection{Backgrounds}

\zhlipsum[11]

\subsection{摸鱼的历史}
\esubsection{History of Underwork}

\zhlipsum[12]

\subsection{摸鱼的历史}
\esubsection{History of Underwork}


\subsubsection{第四级标题}

\paragraph{第五级标题}

\subparagraph{第六级标题}

\subsubparagraph{第七级标题}


公式如下:
\begin{equation}\label{eqn:c1:mdl:constraint_discharge}
    -e^{\max}_\text{dis} \leq a_t \leq e^{\max}_\text{ch}
\end{equation}
上式表示

所以如式\eqref{eqn:c1:mdl:constraint_discharge}所示:
\begin{equation}
    -e^{\max}_\text{dis} \leq a_t \leq e^{\max}_\text{ch}
\end{equation}

最后\footnote{脚注序号“\ding{172},……,\ding{180}”的字体是“正文”,不是“上标”,序号与脚注内容文字之间空1个半角字符,脚注的段落格式为:单倍行距,段前空0磅,段后空0磅,悬挂缩进1.5字符;中文用宋体,字号为小五号,英文和数字用Times New Roman字体,字号为9磅;中英文混排时,所有标点符号(例如逗号“,”、括号“()”等)一律使用中文输入状态下的标点符号,但小数点采用英文状态下的样式“.”。}

\begin{enumerate}
    \item 123
    \item 231421
    \item 124124
\end{enumerate}

\begin{theorem}[勾股定理]
    若 $a,b$ 为直角三角形的两条直角边,$c$ 为斜边,那么 $a^2 + b^2 + c^2.$
\end{theorem}

\begin{proof}
{
    通过...

    所以:
    \begin{equation*}
        G(x, y) = G(y, x).  \qedhere
    \end{equation*}
}
\end{proof}

\begin{proposition}
    所以:
    \begin{equation*}
        G(x, y) = G(y, x).
    \end{equation*}
\end{proposition}


\begin{conjecture}[勾股定理]
    若 $a,b$ 为直角三角形的两条直角边,$c$ 为斜边,那么 $a^2 + b^2 + c^2.$
\end{conjecture}

\begin{axiom}[勾股定理]
    若 $a,b$ 为直角三角形的两条直角边,$c$ 为斜边,那么 $a^2 + b^2 + c^2.$
\end{axiom}

\begin{definition}[勾股定理]
    若 $a,b$ 为直角三角形的两条直角边,$c$ 为斜边,那么 $a^2 + b^2 + c^2.$
\end{definition}




