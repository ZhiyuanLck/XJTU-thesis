% !TeX root = ../main.tex

% 由于需要生成中英目录,本模板提供了 \xLEVEL{}{} 的命令,其中,
%   LEVEL 为 chapter, section, subsection;
%   第一个参数为中文,第二个参数为英文(不论中英文主体)
% 由于目录只包含三级标题,因此四级及以下标题不支持双语,如论文主体为英文写作,请自行修改为英文
% ToC only need top three levels, thus other titles need to be changed to english if you use `english' option. 

\xchapter{绪论}{Introductions}

\xsection{背景}{Backgrounds}

本文档仅提供部分可能用到的示例,系统的学习\LaTeX 用法请参考其他书目,如 \emph{lshort},可通过在 shell(命令行/cmd) 中执行 \clist{texdoc lshort-zh} 获得中文版

\xsection{基本功能}{Basic Functions}

\xsubsection{脚注及其使用}{Footnotes}

脚注\footnote{脚注序号“\ding{172},……,\ding{180}”的字体是“正文”,不是“上标”,序号与脚注内容文字之间空1个半角字符,脚注的段落格式为:单倍行距,段前空0磅,段后空0磅,悬挂缩进1.5字符;中文用宋体,字号为小五号,英文和数字用Times New Roman字体,字号为9磅;中英文混排时,所有标点符号(例如逗号“,”、括号“()”等)一律使用中文输入状态下的标点符号,但小数点采用英文状态下的样式“.”。}\footnote{脚注序号“\ding{172},……,\ding{180}”的字体是“正文”,不是“上标”,序号与脚注内容文字之间空1个半角字符,脚注的段落格式为:单倍行距,段前空0磅,段后空0磅,悬挂缩进1.5字符;中文用宋体,字号为小五号,英文和数字用Times New Roman字体,字号为9磅;中英文混排时,所有标点符号(例如逗号“,”、括号“()”等)一律使用中文输入状态下的标点符号,但小数点采用英文状态下的样式“.”。}\footnote{脚注序号“\ding{172},……,\ding{180}”的字体是“正文”,不是“上标”,序号与脚注内容文字之间空1个半角字符,脚注的段落格式为:单倍行距,段前空0磅,段后空0磅,悬挂缩进1.5字符;中文用宋体,字号为小五号,英文和数字用Times New Roman字体,字号为9磅;中英文混排时,所有标点符号(例如逗号“,”、括号“()”等)一律使用中文输入状态下的标点符号,但小数点采用英文状态下的样式“.”。}可以通过 \clist{\footnote{}}自动编号生成,也可使用\clist{\footnotetex{text}}手动添加脚注

\xsubsection{加粗,斜体与其他字体设置}{Bold, Emph and Other Font Settings}
\xsubsection{文档层级}{Level of this document}

按照学校的要求,正文最多应具有七个层级,分别使用以下命令
按照学校的要求,正文最多应具有七个层级,分别使用以下命令
按照学校的要求,正文最多应具有七个层级,分别使用以下命令
按照学校的要求,正文最多应具有七个层级,分别使用以下命令
按照学校的要求,正文最多应具有七个层级,分别使用以下命令
按照学校的要求,正文最多应具有七个层级,分别使用以下命令
按照学校的要求,正文最多应具有七个层级,分别使用以下命令
按照学校的要求,正文最多应具有七个层级,分别使用以下命令
按照学校的要求,正文最多应具有七个层级,分别使用以下命令
按照学校的要求,正文最多应具有七个层级,分别使用以下命令

\xsection{文档层级}{Level of this document}
\xsubsection{文档层级}{Level of this document}

按照学校的要求,正文最多应具有七个层级,分别使用以下命令
按照学校的要求,正文最多应具有七个层级,分别使用以下命令
按照学校的要求,正文最多应具有七个层级,分别使用以下命令
按照学校的要求,正文最多应具有七个层级,分别使用以下命令
按照学校的要求,正文最多应具有七个层级,分别使用以下命令
按照学校的要求,正文最多应具有七个层级,分别使用以下命令
按照学校的要求,正文最多应具有七个层级,分别使用以下命令
按照学校的要求,正文最多应具有七个层级,分别使用以下命令
按照学校的要求,正文最多应具有七个层级,分别使用以下命令
按照学校的要求,正文最多应具有七个层级,分别使用以下命令

\xsubsection{文档层级}{Level of this document}

按照学校的要求,正文最多应具有七个层级,分别使用以下命令
按照学校的要求,正文最多应具有七个层级,分别使用以下命令
按照学校的要求,正文最多应具有七个层级,分别使用以下命令
按照学校的要求,正文最多应具有七个层级,分别使用以下命令
按照学校的要求,正文最多应具有七个层级,分别使用以下命令
按照学校的要求,正文最多应具有七个层级,分别使用以下命令
按照学校的要求,正文最多应具有七个层级,分别使用以下命令
按照学校的要求,正文最多应具有七个层级,分别使用以下命令
按照学校的要求,正文最多应具有七个层级,分别使用以下命令
按照学校的要求,正文最多应具有七个层级,分别使用以下命令

\xsubsection{文档层级}{Level of this document}

按照学校的要求,正文最多应具有七个层级,分别使用以下命令
按照学校的要求,正文最多应具有七个层级,分别使用以下命令
按照学校的要求,正文最多应具有七个层级,分别使用以下命令
按照学校的要求,正文最多应具有七个层级,分别使用以下命令
按照学校的要求,正文最多应具有七个层级,分别使用以下命令
按照学校的要求,正文最多应具有七个层级,分别使用以下命令
按照学校的要求,正文最多应具有七个层级,分别使用以下命令
按照学校的要求,正文最多应具有七个层级,分别使用以下命令
按照学校的要求,正文最多应具有七个层级,分别使用以下命令
按照学校的要求,正文最多应具有七个层级,分别使用以下命令










\begin{table}[H]
  \begin{tabularx}{\textwidth}{YY}
  \toprule
      层级 & 命令 \\
  \midrule
      0  & \lstinline|\\xchapter{Chs}{Eng}| \\
      1  & \lstinline|\\xsection{Chs}{Eng}| \\
      2  & \lstinline|\\xsubsection{Chs}{Eng}| \\
      3  & \lstinline|\\subsubsection{Chs/Eng}| \\
      4  & \lstinline|\\paragraph{Chs/Eng}| \\
      5  & \lstinline|\\subparagraph{Chs/Eng}| \\
      6  & \lstinline|\\subsubparagraph{Chs/Eng}| \\
  \bottomrule
  \end{tabularx}
\end{table}

示例如下
\subsubsection{第四级标题1}

\paragraph{第五级标题1}

\paragraph{第五级标题2}

\paragraph{第五级标题3}

\subparagraph{第六级标题1}

\subparagraph{第六级标题2}

\subparagraph{第六级标题3}

\subsubparagraph{第七级标题1}

一些文字

\xsection{数字、公式和定理环境}{Equation and Theorem}

\xsubsection{数字与单位}{Numbers and Units}

\xsubsection{公式、矩阵与数学符号}{Equations, Matrix and Mathematical Symbols}

公式如下:
\begin{equation}
    -e^{\max}_\text{dis} \leq a_t \leq e^{\max}_\text{ch}\label{eqn:c1:mdl:constraint_discharge}
\end{equation}
上式表示

所以如式\eqref{eqn:c1:mdl:constraint_discharge}所示:
\begin{equation}
    -e^{\max}_\text{dis} \leq a_t \leq e^{\max}_\text{ch}
\end{equation}

\xsubsection{定理相关}{Theorems}

\begin{theorem}[勾股定理]
    若 $a,b$ 为直角三角形的两条直角边,$c$ 为斜边,那么 $a^2 + b^2 + c^2.$
\end{theorem}

\begin{proof}
    通过\ldots

    所以:
    \begin{equation*}
        G(x, y) = G(y, x).  \qedhere
    \end{equation*}
\end{proof}

\begin{proposition}
  所以:
  \begin{equation*}
      G(x, y) = G(y, x).
  \end{equation*}
\end{proposition}

\begin{conjecture}[勾股定理]
    若 $a,b$ 为直角三角形的两条直角边,$c$ 为斜边,那么 $a^2 + b^2 + c^2.$
\end{conjecture}

\begin{axiom}[勾股定理]
    若 $a,b$ 为直角三角形的两条直角边,$c$ 为斜边,那么 $a^2 + b^2 + c^2.$ 若 $a,b$ 为直角三角形的两条直角边,$c$ 为斜边,那么 $a^2 + b^2 + c^2.$ 若 $a,b$ 为直角三角形的两条直角边,$c$ 为斜边,那么 $a^2 + b^2 + c^2.$
\end{axiom}

\begin{definition}[勾股定理]
    若 $a,b$ 为直角三角形的两条直角边,$c$ 为斜边,那么 $a^2 + b^2 + c^2.$
\end{definition}

\xsection{其他环境}{Other Environments}

\xsubsection{枚举环境}{Enumerates}

\begin{enumerate}
  \item 123
  \item 231421
  \item 124124
\end{enumerate}
