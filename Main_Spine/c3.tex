% !TeX root = ../main.tex

\xchapter{算法与代码}{Algorithm and Code}

\begin{tcolorbox}[colback=red!5!white,colframe=red!75!black]
  学校并未对 算法 或 代码 的排版格式作出限定,因此可以自行修改。
\end{tcolorbox}

\xsection{算法}{Algorithm}

在仔细研究后,本模板使用了基于 \clist[columbiablue]{algorithm2e} 宏包的算法排版工具,而没有使用 \clist[columbiablue]{algorithmicx (algpseudocode)}。同时,需要提醒的是, \clist[columbiablue]{algpseudocode} 在制作竖直定界线方面有一些问题,而能够较好的实现算法自动分段;\clist[columbiablue]{algorithm2e} 则相反,且这两个宏包相互冲突。

算法的具体编写方式请自行查阅宏包,即\clist{texdoc algorithm2e},本模板已预先定义了Do-While循环即\clist{\Do{cond}{code}}。

\begin{texcode}[]{}
  \begin{algorithm}[H]
    \SetAlgoLined
    \KwData{this text}
    \KwResult{how to write algorithm with \LaTeX2e }
  
    initialization\;
    \While{not at end of this document}{
      read current\;
      \eIf{understand}{
        go to next section\;
        current section becomes this one\;
        }{
        go back to the beginning of current section\;
        }
      }
    \caption{How to write algorithms}
  \end{algorithm}
\end{texcode}

\begin{algorithm}[h]
  \caption{Simulation-optimization heuristic}\label{algorithm}
  \KwData{current period $t$, initial inventory $I_{t-1}$, initial capital $B_{t-1}$, demand samples}
  \KwResult{Optimal order quantity $Q^{\ast}_{t}$}
  $r\leftarrow t$\;
  $\Delta B^{\ast}\leftarrow -\infty$\;
  \While{$\Delta B\leq \Delta B^{\ast}$ and $r\leq T$}{$Q\leftarrow\arg\max_{Q\geq 0}\Delta B^{Q}_{t,r}(I_{t-1},B_{t-1})$\;
  $\Delta B\leftarrow \Delta B^{Q}_{t,r}(I_{t-1},B_{t-1})/(r-t+1)$\;
  \If{$\Delta B\geq \Delta B^{\ast}$}{$Q^{\ast}\leftarrow Q$\;
  $\Delta B^{\ast}\leftarrow \Delta B$\;}
  $r\leftarrow r+1$\;}
\end{algorithm}


\begin{algorithm}[H]
  \KwData{this text}
  \KwResult{how to write algorithm with \LaTeX2e }
  initialization\;
  \While{not at end of this document}{
    read current\;
    \Repeat{this end condition}{
      do these things\;
    }
    \eIf{understand}{
      go to next section\;
      current section becomes this one\;
    }{
      go back to the beginning of current section\;
    }
    \Do{this end condition}{
      do these things\;
    }
  }
  \caption{How to write algorithms}
\end{algorithm}



\xsection{导入代码}{Input codes}

从现有文件导入源码,使用 \clist[columbiablue]{lstlisting} 宏包,本模板预先定义了一基本格式 \clist{sty_basic} 以供使用,可以自定义新的样式。

\begin{texcodeonly}
\lstinputlisting[style=sty_basic,
  language=matlab,
  caption={标准粒子群算法},
  label={lst:a_label}]{./Codes/pso.m}
\end{texcodeonly}
\lstinputlisting[style=sty_basic,
                    language=matlab,
                    caption={标准粒子群算法},
                    label={lst:a_label}]{./Codes/pso.m}