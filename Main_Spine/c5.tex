% !TeX root = ../main.tex

\xchapter{非正文部份的要求}{Requirements of other parts}

\xsection{摘要}{Abstracts}

\begin{tcolorbox}[colback=blue!5!white,colframe=blue!75!black,title=摘要内容要求]
  内容一般包括:从事这项研究工作的目的和意义;完成的工作(作者独立进行的研究工作及相应结果的概括性叙述);获得的主要结论(这是摘要的中心内容)。博士学位论文摘要应突出论文的创新点,硕士学位论文摘要应突出论文的新见解。
\end{tcolorbox}

\begin{tcolorbox}[colback=red!5!white,colframe=red!75!black,title=中文摘要主体格式]
  \begin{enumerate}[leftmargin=0.5cm]
    \item 论文摘要由摘要正文、关键词、论文类型、资助申明等部分组成。
    \item 博士学位论文摘要正文为1000字(word)左右,硕士学位论文摘要正文为600字(word)左右。
    \item 摘要中一般不用图、表、化学结构式、非公知公用的符号和术语。
    \item 如果论文的主体工作得到了有关基金资助,应在摘要第一页的页脚处标注:本研究得到某某基金(编号:)资助。(五号)
  \end{enumerate}
\end{tcolorbox}

\begin{tcolorbox}[colback=red!5!white,colframe=red!75!black,title=英文摘要主体格式]
  \begin{enumerate}[leftmargin=0.5cm]
    \item 用词准确,符合语法;
    \item 关键词按相应专业的标准术语写出,尽量从《英语主题词表》中摘选;
    \item 如果论文的主体工作得到了有关基金资助,应用英文在摘要第一页的页脚处标注:本研究得到某某基金(编号:)资助;
    \item 中文摘要和英文摘要均不要求学位申请人及其指导教师签字。
    \item 摘要正文每段开头不空格,每段之间空一行;
  \end{enumerate}
\end{tcolorbox}

\begin{tcolorbox}[colback=blue!5!white,colframe=blue!75!black,title=关键词要求]
  关键词由3~5个组成。关键词应从《汉语主题词表》中摘选,当《汉语主题词表》的词不足以反映主题时,可由申请人设计关键词,但须加注。每一关键词之间用分号分开,最后一个关键词后不打标点符号。由申请人设计的关键词,须在该关键词的右上角标注*,并在该页的页脚处注明“*表示非汉语主题词”。
\end{tcolorbox}

\begin{tcolorbox}[colback=blue!5!white,colframe=blue!75!black,title=论文类型要求]
  论文类型包括:a.理论研究(Theoretical Research);b.应用基础(Application Fundamentals);c.应用研究(Application Research);d.研究报告(Research Report);e.设计报告(Design Report);f.案例分析(Case Study);g.调研报告(Investigation Report);h.产品研发(Product Development);i.工程设计(Engineering Design);j.工程/项目管理(Engineering/Project Management);k.其它(Others)。
\end{tcolorbox}

\xsection{主要符号表}{Glossary}

\begin{tcolorbox}[colback=red!5!white,colframe=red!75!black]
  \begin{enumerate}[leftmargin=0.5cm]
    \item 如果论文中使用了大量的物理量符号、标志、缩略词、专门计量单位、自定义名词和术语等,应将全文中常用的这些符号及意义列出。如果上述符号和缩略词使用数量不多,可以不设专门的主要符号表,但在论文中出现时须加以说明。
    \item 论文中主要符号应全部采用法定单位,特别要严格执行GB3100~3102—93有关“量和单位”的规定。单位名称的书写,可以采用国际通用符号,也可以用中文名称,但全文应统一,不得两种混用。
    \item 缩略词应列出中英文全称。
    \item 主要符号表正文统一左缩进一个字符。
  \end{enumerate}
\end{tcolorbox}


\xsection{攻读学位期间取得的研究成果}{Achievements}

\begin{tcolorbox}[colback=red!5!white,colframe=red!75!black]
  \begin{enumerate}[leftmargin=0.5cm]
    \item 已发表或已录用的学术论文、已出版的专著/译著、已获授权的专利按参考文献格式列出。
    \item 科研获奖,列出格式为:获奖人(排名情况).项目名称.奖项名称及等级,发奖机构,获奖时间.。
    \item 与学位论文相关的其它成果参照参考文献格式列出。
    \item 全部研究成果连续编号编排。
    \item 用于双盲评审的论文,只列出已发表的学术论文的篇名、发表刊物名称,必须隐去各类论文检索号、期号、卷号、页码;专利号;日期等。
  \end{enumerate}
\end{tcolorbox}



\xsection{答辩委员会会议决议}{Decisions}

\begin{tcolorbox}[colback=red!5!white,colframe=red!75!black]
  \begin{enumerate}[leftmargin=0.5cm]
    \item 填写内容应与学位(毕业)审批材料中答辩委员会决议书一致。
    \item 无需签名。
    \item 盲审论文仅保留“答辩委员会会议决议” 标题
  \end{enumerate}
\end{tcolorbox}