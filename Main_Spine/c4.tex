% !TeX root = ../main.tex

\xchapter{参考文献与交叉引用}{Crossreferences}

注意,当引用时出现 [?] 时,请注意查看辅助文件是否正常生成,此种问题不是模板造成的,请自行解决。

\xsection{公式、图表的引用}{Ref equations, tables or figures}

\xsubsection{引用公式}{Ref equations}

引用公式使用 \clist{\eqref{label}} 命令,将引用公式的编号(带括号):

\begin{texcode}[]{}
  如式 \eqref{eqn:c1:diff} 所示
\end{texcode}

\xsubsection{引用其他内容}{Ref other floatings}

其他浮动体引用请使用 \clist{\ref{label}} 命令,将引用其编号(无括号):

\begin{texcode}[]{}
  如图 \ref{subfig:icon} 所示 \\
  如表 \ref{tab:methodcompare} 所示 \\
  如算法 \ref{algorithm} 所示 \\
  如源码 \ref{lst:a_label} 所示
\end{texcode}

\xsection{参考文献}{References}


\begin{tcolorbox}[colback=red!5!white,colframe=red!75!black]
  \begin{enumerate}[leftmargin=0.5cm]
    \item 参考文献的著录格式应符合国家标准GB/T 7714-2015《文后参考文献著录规则》。参考文献中每条项目应齐全。
    \item 参考文献里面标点符号:英文文献用半角,中文文献用全角。
  \end{enumerate}
\end{tcolorbox}

\begin{tcolorbox}[colback=blue!5!white,colframe=blue!75!black]
  \begin{enumerate}[leftmargin=0.5cm]
    \item 文后著录的参考文献务必实事求是。论文中引用过的文献必须著录,未引用的文献不得出现。应遵循学术道德规范,避免涉嫌抄袭、剽窃等学术不端行为。
    \item 参考文献一般应是作者亲自考察过的对学位论文有参考价值的文献,除特殊情况外,一般不应间接引用。
    \item 参考文献应有权威性,要注意引用最新的文献。
    \item 参考文献的数量:
    博士学位论文,一般应在80篇以上,其中,期刊文献60篇以上,国外文献30篇以上,均以近5年的文献为主。
    硕士学位论文,一般应在30篇以上,其中,期刊文献不少于20篇,国外文献不少于10篇,均以近5年的文献为主。
  \end{enumerate}
\end{tcolorbox}

模板使用 biblatex 编译参考文献,默认采用顺序编码制,有多种引用参考文献的方式,请自行查阅 biblatex-gbt-7715-2015 宏包,即 \clist{texdoc biblatex-gbt-7714-2015}。

在导言区使用\clist{\addbibresource{path/to/.bib file}}添加参考文献数据库,默认参考文献数据库位于 ./References/ 下。

这里主要举例两种引用方式,使用 \clist{\cite{ref}} 的引用结果作为上标出现在正文中,使用 \clist{\parentcite{ref}} 的引用结果作为正文内容(一个名词)出现在正文中。

\begin{texcode}[]{}
  这是一个比较常见的问题 \cite{barella_situ_2021}。可以发现 xxxx \cite{atta_enhanced_2021}。\\

  文献\parencite{张燕2013电气自动化在电气工程中的应用探讨} 提出了 xxx,文献 \parencite{黄雪芳2012探讨电气工程中自动化技术的应用} 则提出了 yyy。
\end{texcode}
